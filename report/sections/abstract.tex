\begin{abstract}
% What is the problem?
% How did you solve it?
% What are the results?
% Conclusion
In 2014, Kaggle organized the Galaxy Zoo challenge to analyze images of galaxies from the Sloan Digital Sky Survey in order to automate the tagging of morphological attributes. For a machine to automatically learn the features directly from data, a deep convolutional neural network was implemented based on an existing architecture that placed highly in prestigious image classification competition a number of years prior. The dataset consisted of \numprint{79975} test images and \numprint{61578} training images of which were cropped and downsampled for training. The model was trained with a variety of batch sizes and input formats. These input formats included further dimensionality reductions, data augmentation, and color space conversions. One of the methods ultimately attained a Top 100 score on the Kaggle leaderboards. Further advances can be done to improve the performance of the model by tuning other hyperparameters such as learning rate and number of epochs.
\end{abstract}
