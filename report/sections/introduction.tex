\section{Introduction}

Why do we study galaxies?

Why does it need to be automated?

As a consequence, astronomy is nowadays one of the main research fields in the big data context (\citeauthor{microsoft-galaxies}).

Where does this data come from?

Our latest galaxy images come from the Dark Energy Camera Legacy Survey (DECaLS). Because it uses a larger telescope, DECaLS is 10 times more sensitive to light than the survey that supplied images to the first iteration of Galaxy Zoo, the Sloan Digital Sky Survey. That means that we can see more detail. (https://www.zooniverse.org/projects/zookeeper/galaxy-zoo/about/research)

Why is machine/deep learning a suitable candidate solution?

Machine learning methods are rapidly becoming a go-to tool for automating data-intensive processes which normally require days or weeks of human processing. Due to the incomprehensible size of the Universe, classifiying celestial bodies is an excellent candidate for the application machine learning.

% Finally the economic incentive. By using machine learnig for this task we are freeing up the man-hours of dozens if not hundreds of astrophysics PhD students whose sole existance is to categorize galaxies, quasars, and red giants for their thesis advisors. In this manner we follow the footsteps of Bertrand Russel who advocated for shorter working hours in American industrial revolution of the 20th century \cite{russell-idleness}. Today, thanks to the increasing plurality of machine learning tools we are again questioning what is a necessary weekly hourly allocation of work.